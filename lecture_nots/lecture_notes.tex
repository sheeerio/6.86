\documentclass{scrartcl}
\usepackage[sexy]{evan}
\usepackage{amsmath}

\title{MIT 6.86x}
\author{Gunbir Singh Baveja}

\begin{document}
\maketitle

\section{Lecture 1}
\paragraph{What is machine learning?}
As a discipline, it aims to design and apply computer 
programs that \textit{learn} from experience (i.e., data) 
for the purpose of modeling, prediction, or control.

\begin{remark}
Machine learning as a discipline aims to design, understand, and apply computer programs that learn from experience (i.e. data) for the purpose of modelling, prediction, and control. We will start with prediction as a core machine learning task.

There are many types of predictions that we can make. We can predict outcomes of events that occur in the future such as the market, weather tomorrow, the next word a text message user will type, or anticipate pedestrian behavior in self driving vehicles, and so on.

We can also try to predict properties that we do not yet know. For example, properties of materials such as whether a chemical is soluble in water, what the object is in an image, what an English sentence translates to in Hindi, whether a product review carries positive sentiment, and so on.

\end{remark}

\paragraph{Supervised learning}
It is easier to express tasks in terms of examples of what 
you want (rather than how to solve them). E.g., image classification (1k categories)

Rather than specifying the solution directly (hard), we automate the process of finding one based on examples.

\begin{remark}
Common to all these “prediction problems" mentioned on the previously is that it is very hard to write down a solution in terms of rules or code directly, and far easier to provide examples of correct behavior. For example, how would you encode rules for translation, or image classification? It is much easier to provide large numbers of translated sentences, or examples of what the objects are on a large set of images. The ability to learn the solution from examples is what has made machine learning so popular and pervasive.

We will start with supervised learning in this course. In supervised learning, we are given an example (e.g. an image) along with a target (e.g. what object is in the image), and the goal of the machine learning algorithm is to find out how to produce the target from the example.

More specifically, in supervised learning, we hypothesize a collection of functions (or mappings) parametrized by a parameter, from the examples (e.g. the images) to the targets (e.g. the objects in the images). The machine learning algorithm then automates the process of finding the parameter of the function that fits with the example-target pairs the best.

\end{remark}

\paragraph{Types of machine learning}
\begin{itemize}
\item Supervised learning - prediction based on eaxmples of 
correct behavior.
\item Unsupervised learning - no explicit target, only data, goal to model/discover.
\item Semi-supervised learning - supplement limited annotations with unsupervised learning.
\item Active learing - learn to query the examples actually needed for learning.
\item Transfer Learning - how to apply what you have learned from A to B.
\item Reinforcement Learning - learning to act, not just predict; goal to optimize the consequences of actions.
\end{itemize}


\end{document}
